\documentclass{article}
\usepackage[utf8]{inputenc}
\usepackage{enumitem}
\usepackage{tikz-cd}
\usepackage{amsmath}
\usepackage{amssymb}

\newcounter{artCounter}
\setcounter{artCounter}{0}
\newcommand{\showArt}{{
\ifnum \value{artCounter}=0
\emojiA
\fi
\ifnum \value{artCounter}=1
\emojiB
\fi
\ifnum \value{artCounter}=2
\emojiC
\fi
\ifnum \value{artCounter}=3
\comartSmileA
\fi
\ifnum \value{artCounter}=4
\comartSmileB
\fi
\ifnum \value{artCounter}=5
\comartSmileC
\fi
\ifnum \value{artCounter}=6
\comartSmileD
\fi
\ifnum \value{artCounter}=7
\comartSmileE
\fi
\addtocounter{artCounter}{1}
\ifnum \value{artCounter}=8
\setcounter{artCounter}{0}
\fi
}}

\newcommand{\emojiA}{{
$$ \langle \wedge \ \_\  \wedge \rangle $$
}}

\newcommand{\emojiB}{{
$$ \langle Q\ _\perp\ Q \rangle $$
}}

\newcommand{\emojiC}{{
$$ \langle Q\ _{^{\dot{\smile}}}\ Q \rangle $$
}}

\newcommand{\comartSmileA}{{
\begin{center}
\begin{tikzcd}[ampersand replacement=\&]
\& *\arrow[ld]\arrow[rd,"h"] \& \& \& *\arrow[ld,swap,"i"]\arrow[rd] \& \\
* \& \& * \& * \& \& * \\
\\
\\
\& \& \phi \arrow[r] \& \theta \& \&
\end{tikzcd}
\end{center}
}}

\newcommand{\comartSmileB}{{
\begin{center}
\begin{tikzcd}[ampersand replacement=\&]
\& *\arrow[ld]\arrow[rd,"h"] \& \& \& *\arrow[ld,swap,"i"]\arrow[rd] \& \\
* \& \& * \& * \& \& * \\
\\
\\
\& \& \phi \arrow[r, bend right=20] \& \theta \& \&
\end{tikzcd}
\end{center}
}}

\newcommand{\comartSmileC}{{
\begin{center}
\begin{tikzcd}[sep=2em,ampersand replacement=\&]
\& *\arrow[ld]\arrow[rd,"h"] \& \& \& \& *\arrow[ld,swap,"i"]\arrow[rd] \& \\
* \& \& * \& \& * \& \& * \\
\\
\\
\& \& \& \phi \arrow[r, loop] \& \& \&
\end{tikzcd}
\end{center}
}}

\newcommand{\comartSmileD}{{
\begin{center}
\begin{tikzcd}[sep=2em,ampersand replacement=\&]
\arrow[dddd, bend right=30]   \pi\& \& *\arrow[ld]\arrow[rd,"h"] \& \& \& \& *\arrow[ld,swap,"i"]\arrow[rd] \& \& \mathcal{H} \arrow[loop right, distance=5em] \arrow[llllllll, dashrightarrow, bend right=25]\\
\& * \& \& * \& \& * \& \& * \&\\
\\
\\
\arrow[rrrrrrrr, dashrightarrow, bend right=20] \theta\& \& \& \& \phi \arrow[r, loop] \& \& \& \& \mathcal{P} \arrow[uuuu, bend right=30]
\end{tikzcd}
\end{center}
}}

\newcommand{\comartSmileE}{{
\begin{center}
\begin{tikzcd}[ampersand replacement=\&]
\& *\arrow[ld]\arrow[rd,"h"] \& \& \& *\arrow[ld,swap,"i"]\arrow[rd] \& \\
* \& \& * \& * \& \& * \\
\\
\& \phi \arrow[rrr,bend right=50] \& \& \& \theta \&
\end{tikzcd}
\end{center}
}}

\newcommand{\comartSmileF}{{
\begin{center}
\begin{tikzcd}[row sep=3em,column sep=3em,ampersand replacement=\&]
\&\&Sa^\prime\arrow[dddd,shorten=21mm,xshift=8mm,yshift=8mm]\arrow[dddd,shorten=21mm,xshift=-8mm,yshift=8mm]\\
\&\&\&\\
Sa\arrow[rrrr,shorten=5em,yshift=1.5mm,leftrightarrow,bend right]\arrow[rruu]\&\&\upsilon\&\&Ta^\prime\arrow[lluu]\\
\&\&\&\&\\
\&\&Ta\arrow[rruu]\arrow[lluu]
\end{tikzcd}
\end{center}
}}

\title{Notto - Category for the Working Mathematician}
\author{tkunn}
\date{August 2020}

\begin{document}

\maketitle

This is solutions for problems on the second edition of \textit{Category for the Working Mathematician} by S. Mac Lane.

\section{}

\subsection{}

\showArt

\subsection{}

\showArt

\subsection{}

\subsubsection{}

TODO

\subsubsection{}

TODO

\subsubsection{}

TODO

\subsubsection{}

TODO

\subsubsection{}

TODO

\subsection{}

\subsubsection{}

TODO

\subsubsection{}

TODO

\subsubsection{}

TODO

\subsubsection{}

TODO

\subsubsection{}

TODO

\subsubsection{}

TODO

\subsection{}

\subsubsection{}

TODO

\subsubsection{}

TODO

\subsubsection{}

TODO

\subsubsection{}

TODO

\subsubsection{}

TODO

\subsubsection{}

TODO

\subsubsection{}

TODO

\subsubsection{}

TODO

\subsubsection{}

TODO

\subsection{}

Let $U$ be a set that satisfies following conditions:

\begin{enumerate}[label=(\roman*)]
\item $x \in u \in U \Rightarrow x \in U$
\item $(u \in U \wedge v \in U) \Rightarrow (\{u, v\}, \langle u, v \rangle, u \times v \in U)$
\item \begin{enumerate}[label=(\arabic*)]
\item $x \in U \Rightarrow \mathcal{P} x \in U$
\item $x \in U \Rightarrow \bigcup x \in U$
\end{enumerate}
\item $\omega \in U$, where $\omega = \{0,1,2,\dots\}$ is a set of all finite ordinal numbers.
\item If there exists a surjection $f : a \rightarrow b$ and $a \in U$ and $b \subset U$, then $b \in U$.
\end{enumerate}

\subsubsection{}



For all $q \in \prod_i f_i$, we can construct a bijection $r : I \rightarrow q$. $q \subset U$ because $\forall w \in q,\ \exists j \in I,\ w \in f_j \in U$. Since $I \in U$ and $q \subset U$, we can say $q \in U$. Therefore $\prod_i f_i \subset U$.

Let $|f_k| \geq |f_i|$ for all $i$. Then, we can construct a surjection $g : f_k^I \rightarrow \prod_i f_i$. Also, we can construct a surjection $h : X \rightarrow f_k^I$, with X is either $\mathcal{P} f_k$ or $\mathcal{P} I$. As $X \in U$, $g \circ h : X \rightarrow \prod_i f_i$ and $\prod_i f_i \subset U$, we can say $\prod_i f_i \in U$.

\subsubsection{}



\showArt

\begin{enumerate}[label=(\alph*)]
\item We can construct a bijection $g :  I \rightarrow \{f_i\ |\ i \in I\}$, $g(i) = f_i$. As $I \in U$ and $f_i \in U$ for all $i$, $\{f_i\ |\ i \in I\} \in U$. Therefore $ \bigcup_i f_i = \bigcup \{f_i\ |\ i \in I\} \in U$.
\item Because $x \in U$, we have $y \in U$ for all $y \in x$. Therefore we can apply $f : x \rightarrow x, f(i) = i$ to (a) to get $\bigcup x \in U$.

Because $a \in U$ and $b \subset U$, we can apply $f$ to (a) to get $\bigcup_i f_i \in U$. $f$ is surjective, therefore $b = \bigcup_i f_i$. Hence $b \in U$.
\end{enumerate}

\subsection{}

\showArt

\subsection{}

\showArt

\section{}

\subsection{}

\showArt

\subsection{}

\showArt

\subsection{}

\subsubsection{}



\showArt

In this section, $\times_S$ is a product operation for sets, $\times_C$ is for categories, $\times_G$ is for groups and $\times_M$ is for monoids.

\textit{Monoids.} Let $M, N$ be a monoid with object $m, n$ respectively. The only object in $M \times_C N$ is $\langle m, n \rangle$. ... (TODO)

\textit{Groups.} Let $G, H$ be a group with object $a, b$ respectively. $G \times_C H$ ... (TODO)

\textit{Sets.} Let $A, B$ be discrete categories. The set of all objects in $A \times_C B$ is $X = A \times_S B$. The set of all arrows in $A \times_C B$ is $\{\langle f, g \rangle\ |\ a \in A,\ b \in B,\ f \in A(a, a)\ \wedge\ g \in B(b, b)\} = \{\langle \mathrm{id}_A(c_0), \mathrm{id}_B(c_1) \rangle\ |\ c \in X\}$. Therefore  $A \times_C B$ is a discrete category of $X = A \times_S B$.

\subsubsection{}



\showArt

Let P, Q be preorders. $\forall a \forall b\ | \hom(a, b) | \leq 1$ for both P and Q. Therefore, for all $p_1 \in P$, $p_2 \in P$, $q_1 \in Q$, $q_2 \in Q$, $| \hom_{P \times Q}(\langle p_1, q_1 \rangle, \langle p_2, q_2 \rangle) | = | \hom_P(p_1, p_2) \times \hom_Q(q_1, q_2) | \leq 1$. Hence $P \times Q$ is preorder.

\subsubsection{}



\showArt

Let $\mathrm{id}_X(a) : a \rightarrow a$ be identity in category $X$.

Let the set of all object in $C$ be the product set $\prod_i C_i$ and $C(a, b) = \prod_i C_i(a_i, b_i)$. Let $\mathrm{id}_C(c)_i = \mathrm{id}_{C_i}(c_i)$ for all $c \in C$. Now, we prove $C$ has a universal property:

\begin{enumerate}
\item For every $i$ there is a functor $P_i : C \rightarrow C_i$.
\item For every category $B$ such that a functor $G_i : B \rightarrow C_i$ presents for every $C_i$, there is a functor $F : B \rightarrow C$, which makes the following diagram commute.

\begin{center}
\begin{tikzcd}[sep=4em]
B \arrow[d, swap, "F", dashrightarrow] \arrow[dr, "G_i"] \\
C \arrow[r, "P_i"] & C_i
\end{tikzcd}
\end{center}
\end{enumerate}

First, we prove $P_i : C \rightarrow C_i$ exists. Let the object function be $P_i(x) = x_i$. Let the arrow function be $P_i(f) = f_i$. For all object $c \in C$, $P_i(\mathrm{id}_C(c)) = \mathrm{id}_C(c)_i = \mathrm{id}_{C_i}(c_i) = \mathrm{id}_{C_i}(P_i(c))$. For all arrow $f$, $g$ in C, $P_i(g \circ f) = (g \circ f)_i = g_i \circ f_i =  P_i(g) \circ P_i(f)$. Therefore $P_i$ is a functor.

Second, we prove $F : B \rightarrow C$ exists. Let the object function be $F(x)_i = G_i(x)$. Let the arrow function be $F(f)_i = G_i(f)$. For all object $b \in B$, $F(\mathrm{id}_B(b))_i = G_i(\mathrm{id}_B(b)) =  \mathrm{id}_{C_i}(G_i(b)) = \mathrm{id}_{C_i}(F(b)_i)$. Thus $F(\mathrm{id}_B(b)) = \mathrm{id}_{C}(F(b))$. For all arrow $f$, $g$ in B, $F(f \circ g)_i = G_i(f \circ g) = G_i(f) \circ G_i(g)$. Thus $F(f \circ g) = F(f) \circ F(g)$. Therefore $F$ is a functor.

\subsubsection{}



\showArt

In $\mathbf{Matr}_K$, the object set is all positive integers $\{1, 2, 3, ...\} = \omega \setminus \{0\}$. $\mathbf{Matr}_K(n, m)$ is all rectangular matrix on $K$ of shape $m \times n$. Therefore $\mathbf{Matr}_K^{\mathrm{op}}$ has the same objects $\omega \setminus \{0\}$ and  $\mathbf{Matr}_K^{\mathrm{op}}(n, m)$ is all rectangular matrix on $K$ of shape $n \times m$.

\subsubsection{}



\showArt

Let $R_T \subseteq (T \rightarrow \mathbb{R})$ be a ring whose elements are continuous functions from a topological space $T$ to real number. We construct $R_T$ as follows:

\begin{enumerate}
    \item Additive identity. $0_{R_T} : x \mapsto 0$.
    \item Multiplicative identity. $1_{R_T} : x \mapsto 1$.
    \item Addition. $f + g : x \mapsto f(x) + g(x)$.
    \item Multiplication. $f \times g : x \mapsto f(x) \times g(x)$.
\end{enumerate}

Let $X$ and $Y$ be any topological spaces. If we have a continuous function $f : Y \rightarrow X$, we can construct a ring homomorphism $H(f) = h : R_X \rightarrow R_Y$. We define $h(r) = r \circ f$. Then $h(0_{R_X}) = 0_{R_Y}$, $h(1_{R_X}) = 1_{R_Y}$, $(h(s + t))(x) = (s + t)(f(x)) = s(f(x)) + t(f(x))$, $(h(s \times t))(x) = (s \times t)(f(x)) = s(f(x)) \times t(f(x))$. Therefore $H(f)$ is a ring homomorphism.

Now we construct a functor $F : \mathbf{Top}^{\mathrm{op}} \rightarrow \mathbf{Rng}$. Let the object function be $F(A) = R_A$ and the arrow function be $F(g) = H(g^{\mathrm{op}})$. For all arrow $a, b$ in $\mathbf{Top}^{\mathrm{op}}$, $F(b \circ a) = H((b \circ a)^{\mathrm{op}}) = H(a^{\mathrm{op}} \circ b^{\mathrm{op}}) = H(b^{\mathrm{op}}) \circ H(a^{\mathrm{op}}) = F(b) \circ F(a)$. For all topological space $T \in \mathbf{Top}^{\mathrm{op}}$, $F(\mathrm{id}(T)) = H(\mathrm{id}(T)) = \mathrm{id}(R_T)$. Therefore $F$ is a functor and $\overline{F}$ is a contravariant functor on $\mathbf{Top}$ to $\mathbf{Rng}$.

\subsection{}

\subsubsection{}



TODO











\subsubsection{}



\showArt

For any functor $T : X \rightarrow B$, if its object function is $T(a) = b$, its arrow function maps $\mathrm{id}(a)$ to $\mathrm{id}(b)$. Such a functor $T$ is an object of $B^X$.

Let $R, S : X \rightarrow B$ be functors and $\tau$ be a map on an object of X to an arrow in $B$. $(\tau : R\ \dot{\rightarrow}\ S) \Leftrightarrow (\forall x \in X,\ \tau_x(R(x)) = S(x))$. Therefore $\hom(R, S) = \{\tau\ |\ \forall x \in X,\ \tau_x(R(x)) = S(x)\}$. Therefore an arrow on $R$ to $S$ exists iff $(\forall x, y \in X,\ e_R(x, y) \rightarrow e_S(x, y))\ \wedge\ (\forall x \in X,\ \hom(R(x), S(x)) \neq \emptyset)$, where $e_T(x, y) \Leftrightarrow (\exists a \in B, \{x, y\} \subseteq \{ w\ |\ a = T(w) \})$.

\subsubsection{}



\showArt

An object of $\mathbf{Ab}^{\mathbf{N}}$ is a map on $\mathbb{N}$ to $\mathbf{Ab}$. Same as above, $\hom(R, S) = \{\tau\ |\ \forall n \in N,\ \tau_n(R(n)) = S(n)\}$. In other words, a map $\tau : \mathbb{N} \rightarrow (\mathbf{Ab} \rightarrow \mathbf{Ab})$ is an arrow iff, for every $n \in \mathbb{N}$, there is a corresponding group homomorphism $\tau_n$ on $R(n)$ to $S(n)$, and, for every $m \in \mathbb{N}$ such that $R(n) = R(m)$, $S(n) = S(m)$. 

\subsubsection{}



\showArt

Let $R, S : P \rightarrow Q$. Then $R$ and $S$ are objects of $Q^P$. Let $\tau$ be a natural transform $\tau : R\ \dot{\rightarrow}\ S$. $\tau$ is an arrow on $R$ to $S$ in $Q^P$. Since $\tau$ is natural and $P$ is preorder, following diagram is commute for every pair of objects $p, p^\prime \in P$. $a = f(p,\,p^\prime)$, where $f(p, p^\prime)$ is the only arrow on $p$ to $p^\prime$. Since $Q$ is preorder, $\tau p = g(Rp, Sp)$, where $g(Rp, Sp)$ is the only arrow on $Rp$ to $Sp$.

\begin{center}
\begin{tikzcd}
Rp \arrow[d, swap, "{Ra}"] \arrow[r, "{\tau p}"] & Sp \arrow[d, "{Sa}"]\\
Rp^\prime \arrow[r, "{\tau p^\prime}"] & Sp^\prime
\end{tikzcd}
\end{center}

From the two downward arrows in the diagram, we can say that $\mathrm{Im}(R)$ and $\mathrm{Im}(S)$ contain the preorder structure of $P$. There are two functors $P \rightarrow \mathrm{Im}(R)$ and $P \rightarrow \mathrm{Im}(S)$, where $\mathrm{Im}(T)$ is a category from the image of the object function of $T$ and all arrows between any two pairs in the image.

As explained above, $\sigma p = g(Rp, Sp)$ for all $\sigma : R\ \dot{\rightarrow}\ S$. Thus $|\hom(R, S)| \leq |\{\sigma\ |\ \forall p \in P,\ \sigma p = g(Rp, Sp)\}| = 1$. Therefore $Q^P$ is preorder.

\subsubsection{}



\showArt

Let $\mathbf{Fin}$ be a category of all finite sets. The object is every finite set and the arrow is every mapping between every pair of finite sets.

Let $G$ be a finite group. $G$ is a category of only one object. Every arrow $a$ in $G$ has its inverse $a^{-1}$ such that $a \circ a^{-1} = a^{-1} \circ a = \mathrm{id}$.

$\mathbf{Fin}^G$ is a category that have any functor on $G$ to $\mathbf{Fin}$ as objects and any natural transform between two objects as arrows. The group $G$ has only one object $x \in G$, thus any functor $T \in \mathbf{Fin}^G$ map $x$ to a finite set $T(x) \in \mathbf{Fin}$, and endomorphisms of $x$ to endomorphisms of $T(x)$. For any arrow $a$, $b$ in $G$, $T(b \circ a) = T(b) \circ T(a)$. Also, $\mathrm{id} = T(\mathrm{id}) = T(a \circ a^{-1}) = T(a) \circ T(a^{-1})$. Thus any element in the image of the arrow function of $T$ is invertible. Therefore $T$ is a permutation representation of $G$.

Now, let $\tau : R\ \dot{\rightarrow}\ S$ be an arrow in $\mathbf{Fin}^G$. Then, the following diagram commutes for any arrow $a$ in $G$:
\begin{center}
\begin{tikzcd}
Rx \arrow[d, swap, "Ra"] \arrow[r, "\tau x"] & Sx \arrow[d, "Sa"]\\
Rx \arrow[r, "\tau x"] & Sx
\end{tikzcd}
\end{center}
Therefore, $\hom(R, S) = \{\tau\ |\ \forall a,\ \tau x \circ R a = S a \circ \tau x\}$... what does it mean???? TODO

\subsubsection{}
TODO

\subsubsection{}
TODO

\subsubsection{}
TODO

\subsection{}

\subsubsection{}

\showArt

symbols: A B C F S T

We prove there is a bijection $F : \mathbf{Cat}(A \times B, C) \rightarrow \mathbf{Cat}(A, C^B)$.

Let $FT = S$, where $T : A \times B \rightarrow C$ and $S : A \rightarrow C^B$. First, we make the object function of $S$. Given $a \in A$, we make a subcategory $a \subseteq A$, a category of an object $a$ and an arrow $\mathrm{id}(a)$. Trivially, there is a functor $f_a : B \rightarrow a \times B$ ($f_a(b) = \langle a, b \rangle$ for objects, $f_a(b) = \langle \mathrm{id}(a), b \rangle$ for arrows). As $a \times B \subseteq A \times B$, now we define $Sa : B \rightarrow C$, $Sa = T \circ f_a$. Second, we make the arrow function of $S$. Let $Sa = \tau$, where $a : a_1 \rightarrow a_2$ is an arrow in $A$, $\tau : S a_1\ \dot{\rightarrow}\ S a_2$, $\tau b = T\langle a, \mathrm{id}(b) \rangle$. Let $g : b_1 \rightarrow b_2$ is an arrow in $B$. Then $Sa_2g(\tau b_1(Sa_1b_1)) = Sa_2b_2 = \tau b_2(Sa_1g (Sa_1b_1))$. Thus $\tau$ is natural.

Next, we prove for all $S : A \rightarrow C^B$, there exists $T : A \times B \rightarrow C$, such that $FT = S$. First, we prove for the object function of $T$. We define $T\langle a, b \rangle = Sab$. $FTab = T\langle a, b \rangle$, thus $FT = S$. Second, we prove for the arrow function. Let $T\langle a, b \rangle = Sa(\mathrm{dom}(b))$. Then $FTab = Sab$ for any arrow $a$ in $A$, object $b \in B$.

Finally, we prove that for all $T_1,\ T_2 : A \times B \rightarrow C$, if $FT_1 = FT_2$, then $T_1 = T_2$. Let $FT_1 = FT_2$ and $n \in \{1,\ 2\}$. We write $T_=\langle x, y \rangle$ when $T_1\langle x, y \rangle = T_2\langle x, y \rangle$. First, for all $a \in A$, $b \in B$, $FT_nab = T_n\langle a, b \rangle$. Thus $T_=\langle a, b \rangle$. Second, for every object $a \in A$, arrow $b$ in $B$, $FT_nab = T_n\langle \mathrm{id}(a), b \rangle$. For every arrow $a$ in $A$, object $b \in B$, $FT_nab = T_n\langle a, \mathrm{id}(b) \rangle$. Now, let $a : a_1 \rightarrow a_2$ is any arrow in $A$, $b : b_1 \rightarrow b_2$ is any arrow in $B$. Then $T_=\langle \mathrm{id}(a_1), b \rangle$, $T_=\langle a, \mathrm{id}(b_2)\rangle$ and $T_=(\langle \mathrm{id}(a_1), b \rangle \circ \langle a, \mathrm{id}(b_2) \rangle)$. Thus $T_=\langle a, b \rangle$.

\subsubsection{}

\showArt

For any $x \in (A \times B)^C$, there are two functions $G(c) = x(c)_0$ and $H(c) = x(c)_1$. Let the object function be $F(x) = \langle G, H \rangle$, where $F : (A \times B)^C \rightarrow A^C \times B^C$.

For any natural transform $\tau : (A \times B)\ \dot{\rightarrow}\ C$, there are two natural transforms $\tau_0 : A\ \dot{\rightarrow}\ C$ and $\tau_1 : B\ \dot{\rightarrow}\ C$. Let $\tau_0(x) = $ and $\tau_1(x) = $.

Thus we can construct the arrow function $F(\tau) = \langle \tau_0, \tau_1 \rangle$.

\subsubsection{}

\subsubsection{}

\subsubsection{}

\subsubsection{}

\subsubsection{}

\subsubsection{}

\subsection{}

\subsection{}

\subsection{}

\section{}

\section{}

\section{}

\section{}

\section{}

\section{}

\section{}

\section{}

\section{}

\section{}

\end{document}
