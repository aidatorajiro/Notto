\documentclass{article}
\usepackage[utf8]{inputenc}
\usepackage{enumitem}
\usepackage{tikz-cd}
\usepackage{amsmath}
\usepackage{amssymb}

\newcounter{artCounter}
\setcounter{artCounter}{0}
\newcommand{\showArt}{{
\ifnum \value{artCounter}=0
\emojiA
\fi
\ifnum \value{artCounter}=1
\emojiB
\fi
\ifnum \value{artCounter}=2
\emojiC
\fi
\ifnum \value{artCounter}=3
\comartSmileA
\fi
\ifnum \value{artCounter}=4
\comartSmileB
\fi
\ifnum \value{artCounter}=5
\comartSmileC
\fi
\ifnum \value{artCounter}=6
\comartSmileD
\fi
\ifnum \value{artCounter}=7
\comartSmileE
\fi
\addtocounter{artCounter}{1}
\ifnum \value{artCounter}=8
\setcounter{artCounter}{0}
\fi
}}

\newcommand{\emojiA}{{
$$ \langle \wedge \ \_\  \wedge \rangle $$
}}

\newcommand{\emojiB}{{
$$ \langle Q\ _\perp\ Q \rangle $$
}}

\newcommand{\emojiC}{{
$$ \langle Q\ _{^{\dot{\smile}}}\ Q \rangle $$
}}

\newcommand{\comartSmileA}{{
\begin{center}
\begin{tikzcd}[ampersand replacement=\&]
\& *\arrow[ld]\arrow[rd,"h"] \& \& \& *\arrow[ld,swap,"i"]\arrow[rd] \& \\
* \& \& * \& * \& \& * \\
\\
\\
\& \& \phi \arrow[r] \& \theta \& \&
\end{tikzcd}
\end{center}
}}

\newcommand{\comartSmileB}{{
\begin{center}
\begin{tikzcd}[ampersand replacement=\&]
\& *\arrow[ld]\arrow[rd,"h"] \& \& \& *\arrow[ld,swap,"i"]\arrow[rd] \& \\
* \& \& * \& * \& \& * \\
\\
\\
\& \& \phi \arrow[r, bend right=20] \& \theta \& \&
\end{tikzcd}
\end{center}
}}

\newcommand{\comartSmileC}{{
\begin{center}
\begin{tikzcd}[sep=2em,ampersand replacement=\&]
\& *\arrow[ld]\arrow[rd,"h"] \& \& \& \& *\arrow[ld,swap,"i"]\arrow[rd] \& \\
* \& \& * \& \& * \& \& * \\
\\
\\
\& \& \& \phi \arrow[r, loop] \& \& \&
\end{tikzcd}
\end{center}
}}

\newcommand{\comartSmileD}{{
\begin{center}
\begin{tikzcd}[sep=2em,ampersand replacement=\&]
\arrow[dddd, bend right=30]   \pi\& \& *\arrow[ld]\arrow[rd,"h"] \& \& \& \& *\arrow[ld,swap,"i"]\arrow[rd] \& \& \mathcal{H} \arrow[loop right, distance=5em] \arrow[llllllll, dashrightarrow, bend right=25]\\
\& * \& \& * \& \& * \& \& * \&\\
\\
\\
\arrow[rrrrrrrr, dashrightarrow, bend right=20] \theta\& \& \& \& \phi \arrow[r, loop] \& \& \& \& \mathcal{P} \arrow[uuuu, bend right=30]
\end{tikzcd}
\end{center}
}}

\newcommand{\comartSmileE}{{
\begin{center}
\begin{tikzcd}[ampersand replacement=\&]
\& *\arrow[ld]\arrow[rd,"h"] \& \& \& *\arrow[ld,swap,"i"]\arrow[rd] \& \\
* \& \& * \& * \& \& * \\
\\
\& \phi \arrow[rrr,bend right=50] \& \& \& \theta \&
\end{tikzcd}
\end{center}
}}

\title{Notto - Category for the Working Mathematician}
\author{tkunn}
\date{August 2020}

\begin{document}

\maketitle

This is notes for problems on \textit{Category for the Working Mathematician} by S. Mac Lane.

\section{}

\subsection{}

\showArt

\subsection{}

\showArt

\subsection{}

\subsubsection{}

TODO

\subsubsection{}

TODO

\subsubsection{}

TODO

\subsubsection{}

TODO

\subsubsection{}

TODO

\subsection{}

\subsubsection{}

TODO

\subsubsection{}

TODO

\subsubsection{}

TODO

\subsubsection{}

TODO

\subsubsection{}

TODO

\subsubsection{}

TODO

\subsection{}

\subsubsection{}

TODO

\subsubsection{}

TODO

\subsubsection{}

TODO

\subsubsection{}

TODO

\subsubsection{}

TODO

\subsubsection{}

TODO

\subsubsection{}

TODO

\subsubsection{}

TODO

\subsubsection{}

TODO

\subsection{}

Let $U$ be a set that satisfies following conditions:

\begin{enumerate}[label=(\roman*)]
\item $x \in u \in U \Rightarrow x \in U$
\item $(u \in U \wedge v \in U) \Rightarrow (\{u, v\}, \langle u, v \rangle, u \times v \in U)$
\item \begin{enumerate}[label=(\arabic*)]
\item $x \in U \Rightarrow \mathcal{P} x \in U$
\item $x \in U \Rightarrow \bigcup x \in U$
\end{enumerate}
\item $\omega \in U$, where $\omega = \{0,1,2,\dots\}$ is a set of all finite ordinal numbers.
\item If there exists a surjection $f : a \rightarrow b$ and $a \in U$ and $b \subset U$, then $b \in U$.
\end{enumerate}

\subsubsection{}

\textbf{Let $I \in U$, $f : I \rightarrow b $ and $f_i \in U$ for all $i \in I$. Proof $\prod_i f_i \in U$.}

For all $q \in \prod_i f_i$, we can construct a bijection $r : I \rightarrow q$. $\forall w \in q, \exists j \in I, w \in f_j \in U$, hence $q \subset U$. As $I \in U$ and $q \subset U$, we can say $q \in U$. Therefore $\prod_i f_i \subset U$.

Let $|f_k| \geq |f_i|$ for all $i$. Then, we can construct a surjection $g : f_k^I \rightarrow \prod_i f_i$. Also, we can construct a surjection $h : X \rightarrow f_k^I$, with X is either $\mathcal{P} f_k$ or $\mathcal{P} I$. As $X \in U$, $g \circ h : X \rightarrow \prod_i f_i$ and $\prod_i f_i \subset U$, we can say $\prod_i f_i \in U$.

\subsubsection{}

\textbf{(a) Let $I \in U$, $f : I \rightarrow b $ and $f_i \in U$ for all $i \in I$. Proof $\bigcup_i f_i \in U$. \\ (b) Proof that (a) implies following if (i), (ii), (iii)(1), (iv) and (v) holds true: \\ \indent (iii)(2) $x \in U \Rightarrow \bigcup x \in U$. \\ \indent (v) If $f : a \rightarrow b$ is surjective and $a \in U$ and $b \subset U$, then $b \in U$.
}

\begin{enumerate}[label=(\alph*)]
\item We can construct a bijection $g :  I \rightarrow \{f_i\ |\ i \in I\}$, $g(i) = f_i$. As $I \in U$ and $f_i \in U$ for all $i$, $\{f_i\ |\ i \in I\} \in U$. Therefore $ \bigcup_i f_i = \bigcup \{f_i\ |\ i \in I\} \in U$.
\item Because $x \in U$, we have $y \in U$ for all $y \in x$. Therefore we can apply $f : x \rightarrow x, f(i) = i$ to (a) to get $\bigcup x \in U$.

Because $a \in U$ and $b \subset U$, we can apply $f$ to (a) to get $\bigcup_i f_i \in U$. $f$ is surjective, therefore $b = \bigcup_i f_i$. Hence $b \in U$.
\end{enumerate}

\subsection{}

\showArt

\subsection{}

\showArt

\section{}

\subsection{}

\showArt

\subsection{}

\showArt

\subsection{}

\subsubsection{}

\textbf{Show product of categories includes product of monoids, product of groups and product of sets.}

In this section, $\times_S$ is a product operation for sets, $\times_C$ is for categories, $\times_G$ is for groups and $\times_M$ is for monoids.

\textit{Monoids.} Let $M, N$ be a monoid with object $m, n$ respectively. The only object in $M \times_C N$ is $\langle m, n \rangle$. ... (TODO)

\textit{Groups.} Let $G, H$ be a group with object $a, b$ respectively. $G \times_C H$ ... (TODO)

\textit{Sets.} Let $A, B$ be discrete categories. The set of all objects in $A \times_C B$ is $X = A \times_S B$. The set of all arrows in $A \times_C B$ is $\{\langle f, g \rangle\ |\ a \in A,\ b \in B,\ f \in \hom_A(a, a)\ \wedge\ g \in \hom_B(b, b)\} = \{\langle \mathrm{id}_A(c_0), \mathrm{id}_B(c_1) \rangle\ |\ c \in X\}$. Therefore  $A \times_C B$ is a discrete category of $X = A \times_S B$.

\subsubsection{}

\textbf{Proof that product of two preorders is preorder.}

Let P, Q be preorders. $\forall a \forall b\ | \hom(a, b) | \leq 1$ for both P and Q. Therefore, for all $p_1 \in P$, $p_2 \in P$, $q_1 \in Q$, $q_2 \in Q$, $| \hom_{P \times Q}(\langle p_1, q_1 \rangle, \langle p_2, q_2 \rangle) | = | \hom_P(p_1, p_2) \times \hom_Q(q_1, q_2) | \leq 1$. Hence $P \times Q$ is preorder.

\subsubsection{}

\textbf{Let $\{C_i\ |\ i \in I\}$ be a family of categories indexed by a set $I$. Show product $C = \prod_i C_i$, its projections $P_i : C \rightarrow C_i$ and universal property of these projections.}

Let $\mathrm{id}_X(a) : a \rightarrow a$ be identity in category $X$.

Let the set of all object in $C$ be the product set $\prod_i C_i$ and $\hom_C(a, b) = \prod_i \hom_{C_i}(a_i, b_i)$. Let $\mathrm{id}_C(c)_i = \mathrm{id}_{C_i}(c_i)$ for all $c \in C$. Now, we proof $C$ has a universal property:

\begin{enumerate}
\item For every $i$ there is a functor $P_i : C \rightarrow C_i$.
\item For every category $B$ such that a functor $G_i : B \rightarrow C_i$ presents for every $C_i$, there is a functor $F : B \rightarrow C$, which makes the following diagram commute.

\begin{center}
\begin{tikzcd}[sep=4em]
B \arrow[d, "F", dashrightarrow] \arrow[dr, "G_i"] \\
C \arrow[r, "P_i"] & C_i
\end{tikzcd}
\end{center}
\end{enumerate}

First, we proof $P_i : C \rightarrow C_i$ exists. Let the object function be $P_i(x) = x_i$. Let the arrow function be $P_i(f) = f_i$. For all object $c \in C$, $P_i(\mathrm{id}_C(c)) = \mathrm{id}_C(c)_i = \mathrm{id}_{C_i}(c_i) = \mathrm{id}_{C_i}(P_i(c))$. For all arrow $f$, $g$ in C, $P_i(g \circ f) = (g \circ f)_i = g_i \circ f_i =  P_i(g) \circ P_i(f)$. Therefore $P_i$ is a functor.

Second, we proof $F : B \rightarrow C$ exists. Let the object function be $F(x)_i = G_i(x)$. Let the arrow function be $F(f)_i = G_i(f)$. For all object $b \in B$, $F(\mathrm{id}_B(b))_i = G_i(\mathrm{id}_B(b)) =  \mathrm{id}_{C_i}(G_i(b)) = \mathrm{id}_{C_i}(F(b)_i)$. Thus $F(\mathrm{id}_B(b)) = \mathrm{id}_{C}(F(b))$. For all arrow $f$, $g$ in B, $F(f \circ g)_i = G_i(f \circ g) = G_i(f) \circ G_i(g)$. Thus $F(f \circ g) = F(f) \circ F(g)$. Therefore $F$ is a functor.

\subsubsection{}

\textbf{Show opposite of $\mathbf{Matr}_K$.}

In $\mathbf{Matr}_K$, the object set is all positive integers $\{1, 2, 3, ...\} = \omega \setminus \{0\}$. $\hom_{\mathbf{Matr}_K}(n, m)$ is all rectangular matrix on $K$ with shape $m \times n$. Therefore $\mathbf{Matr}_K^{\mathrm{op}}$ has the same objects $\omega \setminus \{0\}$ and  $\hom_{\mathbf{Matr}_K^{\mathrm{op}}}(n, m)$ is all rectangular matrix on $K$ with shape $n \times m$.

\subsubsection{}

\textbf{Show that the ring of real continuous functions on a topological space is the object function of a contravariant functor from $\mathbf{Top}$ to $\mathbf{Rng}$.}

Let $R_T \subseteq (T \rightarrow \mathbb{R})$ be a ring whose elements are continuous functions from a topological space $T$ to real number. We construct $R_T$ as follows:

\begin{enumerate}
    \item Additive identity. $0_{R_T} : x \mapsto 0$.
    \item Multiplicative identity. $1_{R_T} : x \mapsto 1$.
    \item Addition. $f + g : x \mapsto f(x) + g(x)$.
    \item Multiplication. $f \times g : x \mapsto f(x) \times g(x)$.
\end{enumerate}

Let $X$ and $Y$ be any topological spaces. If we have a continuous function $f : Y \rightarrow X$, we can construct a ring homomorphism $H(f) = h : R_X \rightarrow R_Y$. We define $h(r) = r \circ f$. Then $h(0_{R_X}) = 0_{R_Y}$, $h(1_{R_X}) = 1_{R_Y}$, $(h(s + t))(x) = (s + t)(f(x)) = s(f(x)) + t(f(x))$, $(h(s \times t))(x) = (s \times t)(f(x)) = s(f(x)) \times t(f(x))$. Therefore $H(f)$ is a ring homomorphism.

Now we construct a functor $F : \mathbf{Top}^{\mathrm{op}} \rightarrow \mathbf{Rng}$. Let the object function be $F(A) = R_A$ and the arrow function be $F(g) = H(g^{\mathrm{op}})$. For all arrow $a, b$ in $\mathbf{Top}^{\mathrm{op}}$, $F(b \circ a) = H((b \circ a)^{\mathrm{op}}) = H(a^{\mathrm{op}} \circ b^{\mathrm{op}}) = H(b^{\mathrm{op}}) \circ H(a^{\mathrm{op}}) = F(b) \circ F(a)$. For all topological space $T \in \mathbf{Top}^{\mathrm{op}}$, $F(\mathrm{id}(T)) = H(\mathrm{id}(T)) = \mathrm{id}(R_T)$. Therefore $F$ is a functor and $\overline{F}$ is a contravariant functor from $\mathbf{Top}$ to $\mathbf{Rng}$.

\subsection{}

\subsubsection{}

\textbf{Show that for any ring $R$, $R\textbf{-}\mathbf{Mod}$ is a full subcategory of $\mathbf{Ab}^R$.}

TODO

% First, we prove for any objects $A, B \in R\textbf{-}\mathbf{Mod}$, a functor $f : A \rightarrow B$ exists. Let $f(a) = 0$. Then for any $r \in R$ and $a \in A$, $f(r \times a) = r \times f(a)$. Also, for any $a_1 \in A$ and $a_2 \in A$, $f(a_1 + a_2) = f(a_1) + f(a_2)$. Thus $f$ is an arrow on $A$ to $B$.

%We define a functor $F : R\textbf{-}\mathbf{Mod} \rightarrow \mathbf{Ab}^R$ by $F() = $.

%We proof $F$ is full.

% First, we can see $R\textbf{-}\mathbf{Mod}$ as a subcategory of $\mathbf{Ab}^R$. 

% Second, we prove for any abelian groups $A, B \in R\textbf{-}\mathbf{Mod}$, $\hom_{R\textbf{-}\mathbf{Mod}}(A, B)=\hom_{\mathbf{Ab}^R}(A, B)$.

\subsubsection{}

\textbf{For a finite discrete category $X$, describe $B^X$.}

For any functor $T : X \rightarrow B$, if its object function is $T(a) = b$, its arrow function only maps $\mathrm{id}(a)$ to $\mathrm{id}(b)$. Such a functor $T$ is an object of $B^X$.

Let $R, S : X \rightarrow B$ be functors and $\tau$ be a map on an object of X to an arrow in $B$. $(\tau : R\ \dot{\rightarrow}\ S) \Leftrightarrow (\forall x \in X,\ \tau_x(R(x)) = S(x))$. Therefore $\hom(R, S) = \{\tau\ |\ \forall x \in X,\ \tau_x(R(x)) = S(x)\}$. Therefore an arrow on $R$ to $S$ exists iff $(\forall x, y \in X,\ e_R(x, y) \rightarrow e_S(x, y))\ \wedge\ (\forall x \in X,\ \hom(R(x), S(x)) \neq \emptyset)$, where $e_T(x, y) \Leftrightarrow (\exists a \in B, \{x, y\} \subseteq \{ w\ |\ a = T(w) \})$.

\subsubsection{}

\textbf{Let $\mathbf{N}$ be a discrete category of natural numbers. Describe $\mathbf{Ab}^{\mathbf{N}}$.}

An object of $\mathbf{Ab}^{\mathbf{N}}$ is a map on $\mathbb{N}$ to $\mathbf{Ab}$. Same as above, $\hom(R, S) = \{\tau\ |\ \forall n \in N,\ \tau_n(R(n)) = S(n)\}$. In other words, a map $\tau : \mathbb{N} \rightarrow (\mathbf{Ab} \rightarrow \mathbf{Ab})$ is an arrow iff, for every $n \in \mathbb{N}$, there is a corresponding group homomorphism $\tau_n$ on $R(n)$ to $S(n)$, and, for every $m \in \mathbb{N}$ such that $R(n) = R(m)$, $S(n) = S(m)$. 

\subsubsection{}

\textbf{Let $P$ and $Q$ be preorders. Describe $Q^P$ and show it is a preorder.}

Let $R, S : P \rightarrow Q$. Then $R$ and $S$ are objects of $Q^P$. Let $\tau$ be a natural transform $\tau : R\ \dot{\rightarrow}\ S$. $\tau$ is an arrow on $R$ to $S$ in $Q^P$. Since $\tau$ is natural and $P$ is preorder, following diagram is commute for every pair of objects $p, p^\prime \in P$, where $f(p, p^\prime)$ is the only arrow on $p$ to $p^\prime$. Since $Q$ is preorder, $\tau x = g(Rx, Sx)$, where $g(Rx, Sx)$ is the only arrow on $Rx$ to $Sx$.

\begin{center}
\begin{tikzcd}[sep=9em]
Rp \arrow[d, "{Rf(p,\,p^\prime)}"] \arrow[r, "{\tau p\,=\,g(Rp,\,Sp)}"] & Sp \arrow[d, "{Sf(p,\,p^\prime)}"]\\
Rp^\prime \arrow[r, "{\tau p^\prime\,=\,g(Rp^\prime,\,Sp^\prime)}"] & Sp^\prime
\end{tikzcd}
\end{center}

From the two downward arrows in the diagram, we can say that $\mathrm{Im}(R)$ and $\mathrm{Im}(S)$ contain the preorder structure of $P$. There are two functors $P \rightarrow \mathrm{Im}(R)$ and $P \rightarrow \mathrm{Im}(S)$, where $\mathrm{Im}(T)$ is a category from the image of the object function of $T$ and all arrows between any two pairs in the image.

As explained above, $\sigma p = g(Rp, Sp)$ for all $\sigma : R\ \dot{\rightarrow}\ S$. Thus $|\hom(R, S)| \leq |\{\sigma\ |\ \forall p \in P,\ \sigma p = g(Rp, Sp)\}| = 1$. Therefore $Q^P$ is preorder.

\subsubsection{}

\textbf{Let $\mathbf{Fin}$ be a category of all finite sets and $G$ be a finite group. Describe $\mathbf{Fin}^G$.}



\subsection{}

\subsection{}

\subsection{}

\subsection{}

\section{}

\section{}

\section{}

\section{}

\section{}

\section{}

\section{}

\section{}

\section{}

\section{}

\end{document}
